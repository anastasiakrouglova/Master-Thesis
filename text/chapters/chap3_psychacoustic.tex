\chapter{Psychoacoustics}
\label{chap:Psychoacoustics}
This chapter will introduce several important concepts in the perception of sound, which will play an important role in the interpretation of resonance spectrograms (Chapter \ref{chap:resonances}) and the creation of hierarchies (Chapter \ref{chap:architecture}) in our knowledge representation. 
Psychoacoustics is the scientific field that studies the human perception of sound and the psychological responses associated with it. We will discuss the fundamental difference between sensing and perceiving in this chapter and explore how inference influences our perception of sound and can create new (unwanted) tones.


\section{Sensing and Perceiving}
Many different descriptions of the distinction between sensing and perceiving were given throughout history.
Aristotle described perception as an act of self-consciousness, representing a reflective self-awareness of our perceptual actions (\cite{kosman_perceiving_1975}).
Another commonly accepted explanation is that learning influences perception but not sensation. This means that perception can vary significantly based on what has been learned over different occasions, while sensitivity remains relatively constant over time (unless temporal changes in sensitivity are established) (\cite{obrien_epistemology_nodate}).  
\begin{marginfigure}
\centering
\includesvg[pretex=\fontsize{8pt}{10pt}, width=0.8\linewidth]{images/music/triangle.svg}
\vspace{0.5cm}
\caption{A Kanizsa triangle illustrating the difference between sensing and perceiving through the formation of an illusionary triangle from incomplete circles.}
\label{triangle} 
\end{marginfigure} 
Figure \ref{triangle} visually illustrates the distinction between sensing, which involves receiving external information from three circles with triangular gaps, and perception, which involves the inference of additional information from the triangle between the circles with gaps (\cite{kellman_theory_1991}).


\section{Perception of Sound}
\subsection{Missing Fundamental}
The phenomenon of the missing fundamental is the perception of the fundamental frequency when it is not physically present in the original sound. This perception occurs because the auditory cortex, the region of the brain responsible for sound processing, interprets repetitive patterns of the overtones that are present (\cite{smith_human_1978, zatorre_finding_2005, schneider_structural_2005}). The model of pitch perception, which will be discussed in Chapter~\ref{chap:resonances}, captures this missing fundamental effect (\cite{homer_modelling_2023}).

\subsection{Combination Tone}

\begin{marginfigure}
\centering
\vspace{0.5cm}
\includesvg[pretex=\fontsize{8pt}{10pt}, width=1\linewidth]{images/music/differenceTone.svg}
\caption{Visualization of difference tones, inspired by the book "For the Contemporary Flutist" (\cite{offermans_for_nodate}).}
\label{differenceTone}  \vspace{3cm}
\end{marginfigure}


A combination tone refers to the psychoacoustic phenomenon where an additional tone or tones are perceived when two actual musical tones are played simultaneously (\cite{hosch_perception_2023}). There are two types of combination tones: difference tones and summation tones. Difference tones are generated by subtracting the frequency of one tone from the frequency of another tone, as shown in Figure \ref{differenceTone}. On the other hand, summation tones are produced by adding the frequencies of the two tones together.


\subsubsection{Difference Tone}
Difference tones are commonly observed when a harmonic series produces a fifth between notes. This phenomenon can also be heard in a room with sufficient reverberation, where the echo of the initial sound interferes with the live sound. Another situation where difference tones can occur is with ethnic flutes combined with big drums (\cite{offermans_for_nodate}). Figure \ref{sampleDifferenceTone} shows a sample excerpt wherein two flutes are playing together, and visualizes the appearing difference tones of this performance.

\begin{figure}[h]
\centering
\includesvg[width=1\linewidth]{images/music/sampleDifferenceTone.svg}
\caption{Sample of a piece for "For the Contemporary Flutist" illustrating difference tones (\cite{offermans_for_nodate}). }
\label{sampleDifferenceTone}  
\end{figure}






The well-known quote by W. A. Mozart, "What's worse than the sound of a flute? Two flutes.", gains a stronger meaning now in the context of difference tones. While the phenomenon is not exclusive to wind instruments, Mozart clearly expressed his dislike for this additional sound artifact in his compositions. However, setting aside Mozart's disfavor towards flutes, difference tones can be seen as an immense extension to the sound of an orchestra, where the sound of multiple instruments interferes with each other. 



\section{The Fundamental and (Non)harmonic Overtones}
\begin{marginfigure}
\centering
\includesvg[pretex=\fontsize{8pt}{10pt}, width=1\linewidth]{images/music/overtonestring.svg}
\vspace{0.3cm}
\caption{A fundamental and its three harmonic overtones.}
\label{harmonicMusic} 
\vspace{1cm}
\end{marginfigure}
A harmonic can be defined as one of the components of the harmonic series, which represents a collection of frequencies that are (nearly) positive integer multiples of a single fundamental frequency. Most real-valued signals, except pure sine waves, consist of a fundamental frequency (the first harmonic) and overtones (higher harmonics), which are sinusoidal components at integer multiples of the fundamental frequency. However, real-valued signals can also contain non-harmonic overtones, which do not follow the harmonic series (\cite{young_inharmonicity_1954}). In Figure \ref{nonharmonicMusic}, the harmonics are depicted with a blue line, while the non-harmonics are shown with an orange line. Organic sounds produced by instruments like guitars and pianos typically include both harmonic and non-harmonic overtones, as the vibrations of metal, wood, and membranes generate non-harmonic overtones, which contribute to the timbre of a sound (\cite{mcadams_perceptual_2019}).

\begin{marginfigure}
\centering
\includesvg[pretex=\fontsize{8pt}{10pt}, width=1\linewidth]{images/music/inharmonics.svg}
\vspace{0.1cm}
\caption{This image depicts the distinction between harmonic overtones, represented by the green lines, and non-harmonic overtones, represented by the blue lines.}
\label{nonharmonicMusic} 
\vspace{0.5cm}
\end{marginfigure}


\subsection{Structualism}
\label{subsec:structualism}
According to the theory of structuralism, everyday perception is composed or constructed from basic \textit{sensations}. Psychologists such as Edward Bradford Titchener, who practiced introspection, developed a systematic method to experimentally deconstruct percepts and identify their \textbf{constituent elements} in order to understand the underlying structure of perception (\cite{hatfield_objectifying_2015}).
In the context of hearing, various artifacts emerge during the transition from sensing to perceiving. Artifacts such as difference tones and missing fundamentals are just a few examples. Thus, if percepts are syntheses of simpler elements, the question arises whether these elements can be experienced and what they would be in that case. This forms the fundamental philosophy behind the multi-hierarchical structure of constituents developed by Nicholas \textcite{harley_abstract_2020}.


\section{Summary}
This part is a transition to our cognitive approach towards musical analysis and elaborated on the difference between sensing and perceiving. We addressed the psychoacoustic phenomenon where additional tones are perceived due to inference. We introduced the basic concepts of the fundamental tone and non-harmonic overtones. Later in this thesis, we will extract the fundamental and harmonic overtones of musical performances and group the constituent elements of a fundamental together in a knowledge representation. 





