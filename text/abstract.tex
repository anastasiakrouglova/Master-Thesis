\chapter*{Abstract}
\label{ch:abstract}
\addcontentsline{toc}{chapter}{\nameref{ch:abstract}}


% CONTEXT AND PROBLEM STATEMENT
The extraction and formation of musical structures through the analysis of complex auditory scenes is a challenging task in signal processing and machine learning. Musical analysis includes multiple open subtasks to be resolved, such as multi-pitch estimation, musical note tracking and multi-pitch streaming.
% OBJECTIVES AND SCOPE
The main goal of this thesis is to create a framework for the multipurpose description and evaluation of music, allowing inference from different subtasks and a general improvement in the learnability of machine learning models.
% METHODOLOGY AND APPROACH
This was achieved by investigating into the implementation of a coherent structure between a spectral analysis of resonances and a type-based knowledge representation in the musical domain, forming an analogy to the perception, cognition and knowledge representation of human intelligence.
% RESULTS AND FINDINGS
We created pitch-based hierarchies formed through density-based clustering techniques in our self-defined hierarchical structure for the definition of musical objects perceived from audio signals.
% CONTRIBUTION TO THE FIELD
Our multipurpose framework for musical analysis has a methodological contribution to various practical applications due to its precision and ability to deal with overlapping sound events, which is one of the key challenges in music signal processing.
% IMPLICATIONS AND SIGNIFICANCE
Approaching this problem through a cognitive perspective has a significant impact on the way machine learning is performed nowadays, due to the possibility of model inference for various subtasks in machine learning. 
Our software also contributes to long-term prospective of explainable modelling and can be used in other early related fields, including speech recognition.
% CONCLUSION
Overall, this thesis bridges the gap between human intelligence and machine learning through the development of a framework for knowledge representation and the recognition of musical objects in a resonance spectrum.