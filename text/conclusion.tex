\chapter*{Conclusion}
\label{ch:conclusion}
\addcontentsline{toc}{chapter}{\nameref{ch:conclusion}}


% SUMMARIZE RESEARCH OBJECTIVES
The aim of this thesis was to create a multipurpose cognitive framework for musical analysis. 
% RECAP METHODOLOGY
We modelled human-like perception with the discrete resonance spectrum, grouped this information with cognitive models and structured the clusters in a type-based knowledge representation.
% RECAP FOUND
We extracted musical structures from audio files and stored them in a hierarchical structure. This way, a bidirectional link between knowledge and data was obtained.
% SIGNIFICANCE
Our methodology allows us to infer knowledge from different methods and build a system for a long-term perspective. 
% MY CONTRIBUTION
Moreover, by using a cognitive clustering-approach, one of the fundamental music transcription challenges of overlapping tones has been resolved.
% CONCLUSION
To conclude, we created a tool that gives a range of possibilities in different subtasks of musical analysis and can push this research domain further by allowing inference between various machine learning models.